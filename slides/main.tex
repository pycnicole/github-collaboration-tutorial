\documentclass[12pt]{beamer}
\usepackage[utf8]{inputenc}
\usepackage{hyperref}
\usepackage{graphicx}
\usepackage{comment}

\usetheme{Madrid}
\usecolortheme{beaver}
\setbeamertemplate{enumerate items}[default]
\AtBeginSection[]{
  \begin{frame}
  	\vfill\centering
  	\usebeamerfont{title}\insertsectionhead\par%
  	\vfill
  	% \vspace{8pt}
  	% \hrule
  \end{frame}
}

\title[Git Tutorial]{Git \& GitHub: A Beginner's Guide for Research Collaboration}
\author{Chiatse Wang}
\institute[ISS AS]{Institute of Statistical Science, Academia Sinica}
\date{April 24, 2025}

\begin{document}

\begin{frame}
  \titlepage
\end{frame}

\begin{frame}
  \frametitle{Main Takeaways}
  \begin{itemize}
    \setlength\itemsep{1em}
    \item Learn how to manage projects and collaborate effectively using GitHub.
    \item Explore a real-world scenario through a hands-on demo.
    \item All materials are available at: \url{https://github.com/chiatsewang/github-collaboration-tutorial}
  \end{itemize}
\end{frame}

\begin{frame}{Outline}
  \tableofcontents
\end{frame}

\section{Introduction}

% Introduction
\begin{frame}
  \frametitle{Introduction}
  \begin{itemize}
    \setlength\itemsep{1em}
    \item Git is a distributed version control system.
    \item Cloud-based Git services (e.g., GitHub, GitLab, Bitbucket) are used for collaboration and backup.
  \end{itemize}
\end{frame}

% Version control
\begin{frame}
  \frametitle{Version Control}

  Version control maintains a history of changes in code and data.
  \begin{itemize}
    \setlength\itemsep{1em}
    \item No more file or projects like \texttt{code\_final\_v2.py} or \texttt{proj\_20250425}.
  \end{itemize}
  
  \vspace{1em}
  \textbf{Why is version control important?}
  \vspace{1em}

  \begin{enumerate}
    \setlength\itemsep{1em}
    \item Track issues and bugs; backtrack or audit changes when needed.
    \item Manage multiple versions of a project with ease.
    \item Reproduce results consistently and reliably.
    \item Collaborate seamlessly with others.
  \end{enumerate}
\end{frame}

\section{Git Basics}

\begin{frame}
  \frametitle{Git Overview}
  \begin{enumerate}
    \setlength\itemsep{1em}
    \item The concepts of staging and branching are fundamental.
    \item A commit represents a snapshot of your project at a specific point in time.
    \item The basic unit of a Git project is a repository, which contains all commits and branches.
    \item Git tracks changes in files and directories over time.
    \item Merging two branches combines their changes into one.
  \end{enumerate}
\end{frame}



\begin{frame}{Branching in Git}
  \begin{itemize}
      \item A \textbf{branch} is a lightweight movable pointer to a commit.
      \item The default branch is usually \texttt{main} or \texttt{master}.
      \item You can create new branches to develop features independently.
  \end{itemize}
  \pause
  \begin{block}{Common Usage}
  \begin{itemize}
      \item \texttt{git branch feature-x} – create a new branch
      \item \texttt{git switch feature-x} – switch to a branch
      \item \texttt{git merge feature-x} – merge changes into current branch
  \end{itemize}
  \end{block}
  \pause
  \textbf{Why use branches?}
  \begin{itemize}
      \item Isolate development
      \item Parallel feature development
      \item Safe experimentation
  \end{itemize}
\end{frame}



\begin{frame}{Git Basics: Working Directory, Staging Area, Repository}
  \begin{itemize}
      \item \textbf{Working Directory}: Your local project folder. Files here can be edited freely.
      \item \textbf{Staging Area (Index)}: A holding area for changes you want to include in the next commit.
      \item \textbf{Repository (Local)}: A database where Git permanently stores all versions of your project.
  \end{itemize}
  \vspace{1em}
  \begin{block}{Workflow}
  \texttt{working directory} $\rightarrow$ \texttt{staging area} $\rightarrow$ \texttt{repository}
  \end{block}
  \begin{itemize}
      \item Use \texttt{git add} to move changes to staging
      \item Use \texttt{git commit} to record changes to the repository
  \end{itemize}
  \end{frame}

\begin{frame}{Essential Git Commands}
    \begin{itemize}
      \item \texttt{git init}
      \item \texttt{git status}
      \item \texttt{git add}
      \item \texttt{git commit}
      \item \texttt{git log}
      \item \texttt{git diff}
    \end{itemize}
  \end{frame}


\section{GitHub Collaboration Best Practices}

\begin{frame}
  \frametitle{A Good Repository}
  About collaboration.
  Good commit message.
  Branching strategy.
  LICENSE and README.
  Release.
  Issues for discussion and Pull Requests.
  \begin{itemize}
    \item Use a clear and consistent naming convention for branches.
    \item Use meaningful commit messages.
    \item Keep commits small and focused.
    \item Use pull requests for code review and discussion.
    \item Use issues to track bugs and feature requests.
    \item Use tags to mark releases.
    \item Use a README file to document your project.
    \item Use a LICENSE file to specify the terms of use.
    \item Use a .gitignore file to exclude files from version control.
  \end{itemize}
\end{frame}


\begin{frame}{Writing Good Commits}
  \begin{itemize}
    \item Start with a clear subject line with suffix \texttt{[feat]}, \texttt{[fix]}, \texttt{[docs]}, etc.
    \item Optionally add body for details
    \item Use present tense, imperative mood
    \item Example: \texttt{\detokenize{[feat] Add data loader for experiment A}}
  \end{itemize}
\end{frame}

\begin{frame}{README and LICENSE}
  \begin{itemize}
    \item \texttt{README.md}: project overview, usage, setup
    \item \texttt{LICENSE}: choose open source license (MIT, Apache 2.0, GPL...)
    \item Helps others use and contribute to your code
  \end{itemize}
\end{frame}


\begin{frame}{Typical GitHub Workflow}
  \begin{enumerate}
    \item Clone or fork a repository
    \item Create a new branch
    \item Make changes and commit
    \item Push to GitHub
    \item Open a Pull Request
  \end{enumerate}
\end{frame}

\begin{frame}{Branching Strategy}
  \begin{itemize}
    \item \texttt{main} — the stable, production-ready branch
    \item \texttt{dev} — the main development branch
    \item Branch naming convention: \texttt{account\_name/feature/feature-name}, \texttt{bugfix/bug-name}
    \item Feature branches should be short-lived and task-specific
    \item Use \texttt{git merge} or \texttt{git rebase} to integrate changes
  \end{itemize}
\end{frame}

\begin{frame}{What is a Pull Request?}
\begin{itemize}
    \item A \textbf{Pull Request (PR)} is a request to merge one branch into another on GitHub.
    \item Usually from a contributor’s feature branch into the main repository branch.
    \item Enables code review, discussion, and collaboration before merging.
\end{itemize}
\vspace{1em}
\textbf{Typical Workflow}
\begin{enumerate}
    \item Fork a repository
    \item Create a new branch and commit changes
    \item Push changes to your fork
    \item Open a pull request on GitHub
    \item Collaborators review and merge
\end{enumerate}
\end{frame}


\begin{frame}{Pull Request Review Process}
  \begin{itemize}
    \item Reviewers can comment, request changes, or approve the PR.
    \item Use comments to discuss code changes and suggest improvements.
    \item Once approved, the PR can be merged into the main branch.
  \end{itemize}
  \pause
  \textbf{Best Practices}
  \begin{itemize}
    \item Keep PRs small and focused.
    \item Provide context in the PR description.
    \item Respond to feedback promptly.
  \end{itemize}  
\end{frame}

\begin{frame}
  \textbf{Example PR Workflow}

  \begin{itemize}
    \item Create a new branch for your feature.
    \item Make changes and commit them.
    \item Push the branch to GitHub.
    \item Open a PR against the main branch.
    \item Review and address comments.
    \item Merge the PR once approved.
    \item Delete the branch after merging.
  \end{itemize}
\end{frame}

\section{Hands-on Demo}
\begin{frame}{Tutorial Instructions}
    \begin{itemize}
        \item All activities will be performed on GitHub Web Interface — no local Git installation required.
        \item Make sure you're logged in to your GitHub account.
        \item You will be asked to:
        \begin{enumerate}
            \item Fork a repository
            \item Create a branch
            \item Make commits with meaningful messages
            \item Submit a Pull Request (PR)
            \item Respond to review requests
            \item Complete a merge
        \end{enumerate}
        \item Don’t worry — we’ll guide you step-by-step!
    \end{itemize}
    
    \vspace{1em}
    \centering
    \textit{Let’s get hands-on and have fun with version control!}
\end{frame}






\begin{frame}{Resources}
  \begin{itemize}
    \setlenght\itemsep{1em}
    \item \href{https://git-scm.com/doc}{git-scm.com/doc}
    \item \href{https://docs.github.com}{docs.github.com}
    \item \href{https://opensource.guide/}{opensource.guide}
    \item \href{https://choosealicense.com}{choosealicense.com}
    \item \href{https://www.conventionalcommits.org/en/v1.0.0/}{conventionalcommits.org}
  \end{itemize}
\end{frame}

\begin{frame}{Thank You!}
  Questions? \newline
  \small GitHub: \texttt{@chiatsewang} \quad Email: \texttt{chiatsewang@stat.sinica.edu.tw}
\end{frame}

\end{document}
